\usepackage[italian]{babel}
\usepackage[utf8]{inputenc}
\usepackage{fancyhdr}
\usepackage{indentfirst}
\usepackage[T1]{fontenc}
\usepackage[final]{microtype}
\usepackage{newlfont}
\usepackage{graphicx}
\usepackage{caption}
\usepackage{subcaption}
\usepackage{wrapfig}
\usepackage{minted}
\usepackage{amssymb}
\usepackage{amsmath}
\usepackage{latexsym}
\usepackage{amsthm}
\usepackage{csquotes}
\usepackage[sorting=none]{biblatex}
\usepackage[hidelinks]{hyperref}
\usepackage{silence}
\usepackage{xargs}

% \addbibresource{bibliografia/bibliografia.bib}

\hyphenation{} % vanno inserite le parole che latex non riesce a tagliare nel modo giusto andando a capo

\hypersetup{
    pdftitle={Progettazione e Implementazione di Agenti Cognitivi per Simulazioni di Evacuazioni di Folle in Alchemist}
}

% Impaginazione
\textwidth=450pt
\oddsidemargin=0pt
\linespread{1.3}
\pagestyle{fancy}
\addtolength{\headheight}{15pt}
\fancyhead[RO,LE]{\thepage}
\fancyhead[RE]{\small{\it{PROGETTAZIONE E IMPLEMENTAZIONE DI AGENTI COGNITIVI\\PER SIMULAZIONI DI EVACUAZIONI DI FOLLE IN ALCHEMIST}}}
\fancyfoot{}