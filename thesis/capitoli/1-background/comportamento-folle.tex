\section{Il comportamento delle folle}
Lo studio del comportamento collettivo è un ambito sociologico che fonda le proprie radici nelle ricerche dell'antropologo e sociologo francese Gustave Le Bon, autore del libro \enquote{Psicologia delle folle} \cite{Bon1895} pubblicato nel 1895, nel quale egli analizza come l'appartenenza di un individuo ad una conformazione come quella della folla possa inconsciamente causare in esso un attenuamento delle proprie capacità critiche, dovuto alla tendenza a conformarsi alle decisioni scaturite ed alimentate dall'istinto collettivo. Questa \i{teoria del contagio}, come la definisce Le Bon stesso, porta ogni individuo a giustificare le proprie azioni identificandosi non più come singolo, ma come parte costitutiva della folla. \newline
Per spiegare le cause di questo fenomeno Ralph Turner e Lewis Killian, durante la seconda metà degli anni '50, formularono la \i{teoria della norma emergente} \cite{Turner1957}, sottolineando come le decisioni di un ristretto numero di persone all'interno della folla, siano accettate dal resto degli appartenenti ad essa perché i processi di interazione creano involontariamente degli obiettivi comuni che rafforzano determinate scelte inibendone altre, anche se queste avrebbero rispecchiato maggiormente la volontà del singolo individuo. \newline
È un fenomeno quanto mai attuale e tangibile che può essere riscontrato anche quando le persone, non essendo compresenti, non sono fisicamente influenzabili una con l'altra, ma subiscono questo meccanismo tramite l'azione dei mass media e di Internet, come dimostra in maniera eclatante lo scandalo di Cambridge Analytica\n{\url{https://web.archive.org/web/20190911084718/https://www.ilpost.it/2018/03/19/facebook-cambridge-analytica/}}. \newline
Quando questo avviene in presenza di una grande quantità di persone concentrate in uno spazio ristretto, però, le conseguenze possono essere catastrofiche anche se in risposta ad un apparentemente insignificante azione localizzata, che rischia di scatenare una reazione a catena e generare una situazione incontrollabile. \newline
La tendenza a riunirsi per condividere passioni, organizzare manifestazioni, o visitare luoghi di interesse è da sempre un elemento caratteristico dell'uomo e molto probabilmente continuerà sempre ad esserlo; riuscire a prevedere il comportamento delle folle presenti in queste occasioni è perciò di fondamentale importanza.

\subsection{L'evacuazione delle folle}
Una situazione in cui è particolarmente difficile studiare il comportamento di una folla è quella dell'evacuazione. Infatti, se la tendenza a conformarsi alle dinamiche collettive è un istinto naturale di ogni individuo in condizioni normali, nelle situazioni di emergenza, dove a causa di un pericolo imminente entrano in gioco fattori come il panico e il nervosismo, una persona modifica le proprie capacità logiche e decisionali \cite{Helbing2011}. \newline 
Le reazioni si manifestano sia fisicamente, ad esempio con l'aumento della propria velocità, sia psicologicamente, con l'attitudine ad affidarsi e ad emulare gli altri elementi della folla, senza riuscire a realizzare che probabilmente anche loro si trovano nella medesima situazione di non lucidità. Quello che ne deriva è un comportamento non coordinato e svantaggioso per l'obiettivo finale, che per le sua somiglianza con il mondo animale viene definito \enquote{effetto gregge}. La figura \ref{fig:herding} illustra le possibili conseguenze che da esso  potrebbero scaturire. \newline
Riuscire a controllare questi atteggiamenti comporterebbe sicuramente un netto miglioramento dei tempi necessari per la buona riuscita dell'evacuazione; tuttavia è irrealistico pensare di poter obbligare una persona ad agire seguendo uno schema premeditato in una situazione caratterizzata da un forte stress emotivo. \newline
Piuttosto, è doveroso prendere in considerazione anche questi comportamenti, per provare ad analizzare approfonditamente lo scenario di interesse e prevedere come decorreranno i fatti.

\fig[0.75]{herding-behavior.png}{herding}{Illustrazione dell'\enquote{effetto gregge} durante l'evacuazione da una stanza che presenta due uscite}

\subsubsection{Evacuazione da edifici o imbarcazioni}
Uno dei settori in cui il tema dell'evacuazione è di primaria importanza è senza dubbio quello della costruzione di strutture destinate ad ospitare molte persone. Durante la progettazione di un edificio o di una imbarcazione, infatti, ci sono molti fattori che devono essere tenuti in considerazione per ridurre i tempi necessari a permettere ai presenti di lasciare in sicurezza il fabbricato. \newline 
Le principali scelte vertono sul posizionamento e la dimensione delle uscite, ma anche sul dislocamento dei diversi oggetti che, per scopi logistici o di arredamento, devono essere presenti nell'ambiente che si sta progettando e che, in sede di evacuazione, rappresentano degli ostacoli che impediscono di raggiungere lo spazio esterno il prima possibile. \newline
Anche se apparentemente semplici, queste decisioni possono spesso nascondere delle insidie che intuitivamente non riusciamo ad interpretare come tali. \newline 
La presenza di un'uscita più grande e visibile delle altre, ad esempio, può sembrare una buona strategia per avere sempre un punto di riferimento con cui orientarsi mentre, in realtà, può trasformarsi in un fattore limitante, in grado di favorire il già menzionato comportamento del gregge, nel momento in cui tutti i presenti, avendo come unico obiettivo quello di scappare, sono involontariamente portati ad andarvi incontro ignorando la presenza delle altre \cite{Almeida2013}. \newline
Al contempo, ci sono delle soluzioni apparentemente pessime che, in realtà, si dimostrano ottime alla prova dei fatti. Un esempio lampante è quello della presenza di una colonna davanti ad un'uscita, che costringendo la folla a dividersi in due gruppi, diminuirebbe la pressione proveniente dal retro e abbasserebbe i tempi di deflusso di oltre il 50\% \cite{Helbing2002}. Paradossalmente, è proprio per questa sua natura contro-intuitiva e per la falsa percezione di pericolo che susciterebbe, che questo espediente non viene quasi mai adottato. \newline
Un'altra interessante idea progettuale è quella della costruzione di corridoi non rettilinei ma a zigzag, che, in caso di una alta pressione, sono adatti a distribuire in più direzioni la spinta dei pedoni \cite{Helbing2005}.

\subsection{Eventi rilevanti}
L'urgenza di dover attuare delle contromisure al fine di perseguire una gestione controllata del comportamento di una folla, è evidenziata dall'ingente numero di incidenti e disastri a cui l'umanità ha assistito nel corso della sua storia. Solo tra il 1971 e il 2011 è possibile elencarne ben 156 \cite{Soomaroo2012} e questo tipo di situazione è destinato a diventare sempre più frequente, poiché la visibilità ed il raggiungimento del luogo in cui si trova l'evento sono obiettivi sempre più facili da conseguire.

\paragraph{La strage di Hillsborough}
Tra i maggiori palchi scenici che fanno da cornice a questi tragici scenari e che vedono la folla come protagonista, non si possono non menzionare gli stadi calcistici. \newline 
L'esempio più tragico facente parte di questa categoria è quello dell'Hillsborough Stadium di Sheffield il 15 aprile del 1989, che ha visto ben 96 morti e oltre 400 feriti a causa della sottostima del numero di spettatori appartenenti alle due tifoserie rivali\n{\url{https://time.com/4836782/hillsborough-disaster-history-charges/}}. \newline 
Le entrate previste, infatti, non erano idonee alla mole di persone accorse. Per provare a rimediare alla situazione, la polizia pensò di aprire anche l'ingresso che conduceva direttamente al settore centrale della curva, credendo in questo modo di evitare che si creassero disordini negli ingressi predisposti. Quello che ne scaturì, invece, fu un disastro irrimediabile; tutti coloro che erano ancora al di fuori dello stadio, intuendo la presenza di una via per arrivare in tempo, si riversarono all'interno di essa e raggiunsero gli spalti, dove, a causa della mancanza di spazio, i tifosi già presenti furono costretti a provare a scavalcare le recinzioni o a rimanere schiacciati e soffocati, come testimonia la foto in figura \ref{fig:hillsborough}.

\fig[0.8]{hillsborough.jpg}{hillsborough}{Tifosi schiacciati contro le recinzioni dal sovraffollamento}[www.time.com]

\paragraph{I pellegrinaggi alla Mecca}
Ancora più preoccupante è il caso del pellegrinaggio alla Mecca, dove il problema dell'altissima densità di persone si ripresenta costantemente. Nonostante questo cammino sia uno dei cinque pilastri della religione islamica, che ogni anno coinvolge più di due milioni di fedeli, sono più di 5000 le persone che hanno perso la vita per motivi inerenti al sovraffollamento\n{\url{https://en.wikipedia.org/wiki/Incidents\_during\_the\_Hajj}}. \newline
Nell'anno 2015 in particolare, il numero delle vittime ha per la prima volta superato i 2000 individui, anche se le circostanze dell'evento non sono ancora ufficialmente chiarite. Dalle dichiarazioni dei sopravvissuti\n{\url{https://web.archive.org/web/20190810075436/https://www.nytimes.com/interactive/2016/09/06/world/middleeast/2015-hajj-stampede.html}}, trapela che le ragioni del disastro siano imputate all'impatto tra due flussi di fedeli, confluiti all'interno della stessa strada in prossimità del ponte Jamarat (figura \ref{fig:hajj}). Un ingente gruppo ha dovuto intraprendere una deviazione a causa dell'interruzione di una delle rotte predefinite e nel momento in cui la massa si è riversata su un tratto destinato ad un altro gruppo, si è generata un'altissima pressione tra i presenti.

\fig[0.8]{hajj.jpg}{hajj}{Fedeli presenti sul ponte Jamarat per il rituale della \enquote{lapidazione del diavolo}}[www.bbc.com]

\paragraph{Le tragedie di piazza San Carlo e di Corinaldo}
Senza dover andare troppo lontano né geograficamente né temporalmente, è doveroso ricordare anche le tragedie di piazza San Carlo a Torino (3 giugno 2017), in occasione della proiezione della finale di Champions League, e della discoteca \enquote{Lanterna Azzurra} di Corinaldo (8 dicembre 2018), in cui era progrmmato il concerto del rapper \i{Sfera Ebbasta}. \newline
In entrambe le occasioni, l'utilizzo di spray urticante ha generato il panico tra i presenti. Nel primo caso, sono stati oltre 30000 i partecipanti e il tentativo di fuga di gran parte di essi, ha portato alla morte di due persone e ad oltre 1500 feriti; nel secondo, il parapetto ha ceduto a causa dell'ingente numero di persone presenti, causando la caduta di svariate decine di spettatori e provocando 6 decessi.

\subsection{Il divario tra simulazione e realtà}
Le procedure di esercitazione all'evacuazione, pensate allo scopo di verificare l'efficacia delle procedure di sicurezza in caso di emergenza, non sono sempre attuabili. Tale limitazione è dovuta al fatto che, tralasciando le difficoltà organizzative e gli ingenti costi dell'operazione in casi analoghi a quelli sopra citati, sarebbe necessario esporre le persone ai reali pericoli che potrebbero insorgere nell'ambiente. \newline 
Effettuare delle simulazioni solamente ipotizzando il potenziale pericolo, come avviene in quelle a cui tutti abbiamo partecipato almeno una volta a scuola o in un luogo di lavoro, si riduce spesso ad una formalità o addirittura ad un momento di svago, che raramente rispecchia le reazioni che i medesimi protagonisti avrebbero nel caso l'emergenza diventasse reale. \newline
Al fine di realizzare predizioni, quello di cui ci si potrebbe munire è uno strumento che non implichi eventuali rischi nell'utilizzo, sia facilmente adattabile in risposta ai cambiamenti, e permetta la valutazione nel minor tempo possibile del maggior numero di scenari di interesse. \newline
Uno strumento capace di sintetizzare insieme tutti questi requisiti è quello della simulazione computerizzata che, negli ultimi 30 anni, ha rappresentato un'importante risorsa per le nuove scoperte conseguite in questo ambito di ricerca.