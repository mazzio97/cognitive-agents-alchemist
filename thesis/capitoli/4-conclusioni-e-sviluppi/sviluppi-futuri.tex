\section{Sviluppi futuri}
Per migliorare il supporto di Alchemist al problema della simulazione del comportamento delle folle, vengono fornite delle direzioni di ricerca verso le quali il lavoro qui presentato può essere esteso.

\paragraph{Forma dei pedoni}
Attualmente i pedoni vengono tutti rappresentati mediante dei cerchi di dimensione predefinita. Alla stregua di altre caratteristiche come la velocità, il sesso o l'età, anche la forma dovrebbe essere eterogenea, in modo da ricalcare più fedelmente l'occupazione reale dell'agente nello spazio. \newline
Per questo motivo, è necessario definire sia diverse dimensioni in relazione all'età e al sesso del pedone sia delineare la corporatura con delle forme geometriche più simili a quelle che questa presenta in natura. Due possibili alternative per quest'ultimo aspetto, menzionate da Chen nell'analisi delle varie versioni esistenti del modello forze sociali \cite{Chen2017}, sono l'ellisse e la rappresentazione \enquote{tre cerchi}, due dei quali costituiscono le spalle mentre quello al centro la testa.

\paragraph{Ruolo dei pedoni}
Le simulazioni che sono state discusse, vedono come protagonisti pedoni che hanno tutti una conoscenza limitata al loro intorno e a quello che vedono dell'ambiente circostante. \newline 
In realtà, potrebbero esistere anche degli agenti che hanno più familiarità di altri con l'ambiente in cui sono inseriti e
quindi, per muoversi all'interno di esso potrebbero far uso di strategie di esplorazione globale come l'\textit{algoritmo di Dijkstra} o $A^{*}$. \newline 
Un esempio di questa differenziazione è fornito in \cite{Tan2019} dalle tipologie \textit{visitor} e \textit{tenant} e può essere usato come punto di partenza per analizzare questo tema.

\paragraph{Collisioni realistiche}
L'introduzione di ambienti caratterizzati da una componente fisica, per ora, impedisce ad un nodo di andare ad occupare una posizione in cui ne è già presente un altro ma non gestisce le forze in gioco in una potenziale collisione tra essi, che viene solamente prevenuta impedendo lo spostamento. \newline
Una zona ad alta densità, è contraddistinta da una quantità enorme di questi urti scaturiti dal movimento frenetico dei pedoni quindi, riuscire a riprodurli, porterebbe degli importanti miglioramenti per il realismo della simulazione.

\paragraph{Relazioni tra gruppi}
Un gruppo, analogamente ad un pedone, può essere considerato come un’agente della simulazione e, in quanto tale, potrebbe essere influenzato o influire sulle scelte degli altri gruppi presenti nell'ambiente. \newline
Queste possibilità vengono analizzate da Qiu e Xu in \cite{Qiu2010}, dove ad ogni gruppo viene associata una matrice intra-relazionale per definire i rapporti tra ogni possibile coppia di pedoni al suo interno e vi è un’unica matrice inter-relazionale per evidenziare l’influenza reciproca dei diversi gruppi presenti. \newline
Grazie a questa modellazione, non solo è possibile differenziare le attitudini delle varie tipologie di gruppo, ma è anche pensabile delineare la struttura che i vari membri o i vari gruppi devono assumere durante il movimento.

\paragraph{Differenze culturali}
Le caratteristiche di un pedone non sono solo biologiche, mentali o strutturali ma possono avere influssi derivanti anche dalla sua cultura. \newline
La storia, le abitudini e l'ideologia di un paese sono indubbiamente fattori che incidono sul modo di agire di un agente che vive al suo interno e perciò vanno studiati per i diversi popoli esistenti e inclusi nei modelli di evacuazione. \newline
Per avere un'idea più chiara di come inserire questi aspetti in un modello ad agenti, si può partire dal contributo basato su osservazioni empiriche descritto in \cite{Fridman2013}.