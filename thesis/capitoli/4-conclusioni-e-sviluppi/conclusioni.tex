\section{Conclusioni}
Traendo ispirazione dalle idee presentate nel modello IMPACT, le caratteristiche inerenti un agente sono state rielaborate ed introdotte all'interno del simulatore Alchemist, differenziando sulla base di esse tre tipologie di pedone: omogeneo, eterogeneo, continuo. L'utilizzo di questo approccio permette la scelta del livello di dettaglio che si ritiene più opportuno per rappresentare i protagonisti della simulazione. \newline
Al fine di riprodurre i movimenti umani si è optato per l'uso dei comportamenti di steering trattati da Reynolds, forze determinate da scelte localizzate che, se unite tra loro, possono rappresentare dinamiche complesse. Non esistendo una regola generale per combinare questi atteggiamenti, sono state formulate varie strategie per poter rispondere alle diverse esigenze che uno specifico scenario o una particolare situazione possono richiedere. \newline
Data la frequente presenza di gruppi di persone (famiglie, coppie, amici) rispetto a quella di individui solitari, anche questo concetto e le relative dinamiche sono state inclusi nella modellazione degli agenti. \newline
Nonostante la metodologia utilizzata sia principalmente basata sul modello ad agenti, sono stati adottati anche aspetti caratteristici degli automi cellulari e delle forze sociali, in accordo con il suggerimento di Zheng di utilizzare insieme più approcci per sfruttare al meglio i vantaggi di ognuno di essi \cite{Zheng2009}. \newline
L'uso del paradigma orientato agli oggetti e di un linguaggio all'avanguardia come Kotlin, garantiscono delle solide basi per poter ampliare questo lavoro, raffinandolo o estendendolo con altre funzionalità. La natura agnostica all'incarnazione utilizzata, su cui è incentrata la progettazione, permette ampia libertà di scelta su come sfruttare le nuove potenzialità offerte. \newline
Per dimostrare gli effetti dei diversi fattori considerati per modellare i pedoni, sono state eseguite delle simulazioni significative, valutate sia in assenza che in presenza del fenomeno in esame e poi confrontate. Differenze cospicue tra le due versioni sono state riscontrate in tutti gli scenari e i risultati denotano macroscopicamente delle attitudini che, in accordo con le ricerche basate su osservazioni empiriche citate, rispecchiano quelle che le controparti umane degli agenti avrebbero in natura. \newline
A fronte di questi contributi, il simulatore Alchemist possiede i principali elementi necessari a riprodurre l'evacuazione di una folla in svariate situazioni. Ancora la maturità necessaria a poter descrivere esaustivamente circostanze che, per la struttura dell'ambiente in cui sono analizzate o l'ingente numero di aspetti valutati contemporaneamente, si rivelano troppo complesse non è stata raggiunta, ma i primi passi per perseguirla sono stati mossi.