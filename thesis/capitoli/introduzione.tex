Le vicende di piazza San Carlo a Torino e della discoteca \enquote{Lanterna Azzurra} di Corinaldo sono solo degli esempi a noi vicini geograficamente e cronologicamente di come l'evacuazione di una folla possa sfociare in una tragedia se non si considerano adeguatamente: l'afflusso, i potenziali pericoli e il comportamento dei partecipanti. \newline
La simulazione computerizzata rappresenta uno dei più potenti strumenti a disposizione dell'uomo per prevenire questo genere di situazioni, evitando errori che possono irrompere in una vastissima quantità di casistiche, dall'organizzazione di un evento alla costruzione di un edificio. \newline
Gli ultimi trent'anni hanno visto la nascita di moltissimi modelli che, pur derivando la loro formulazione da teorie molto diverse, perseguono tutti il medesimo scopo di riprodurre le azioni di un notevole numero di persone. Tra le varie metodologie di simulazione presenti, quella basata sugli agenti, esseri autonomi in grado di interagire tra loro, è stata scelta come punto di partenza, data la sua flessibilità e naturalezza nel descrivere un sistema costituito da molte componenti differenti come quello in esame. \newline
Mentre notevoli sviluppi sono stati conseguiti in merito alla descrizione delle varie forze fisiche in gioco, meno esplorati sono i temi relativi alle capacità cognitive e relazionali delle persone. \newline 
Tali aspetti sono essenziali da considerare per ambire ad un risultato realistico e non possono essere trascurati nello studio del comportamento di una folla, soprattutto nei casi di emergenza, dove la loro importanza è ulteriormente amplificata. É in queste occasioni, infatti, che fenomeni quali la trasmissione del panico e l'appartenenza ad un gruppo costituiscono le principali cause delle decisioni prese da ogni individuo. \newline
Focalizzando l'attenzione su questi concetti, il lavoro di tesi qui presentato ha come obiettivo quello di inserire all'interno del simulatore stocastico Alchemist gli elementi caratterizzanti l'evacuazione di una folla in una accezione generale, senza concentrarsi su nessuno scenario specifico, in modo da fornire gli strumenti necessari per ricreare svariate situazioni di interesse. \newline
Il ruolo di agente della simulazione è assunto da ogni singolo pedone; di conseguenza, parte integrante del progetto è stata quella di attribuire ad esso la facoltà di muoversi all'interno dell'ambiente considerando la combinazione di diverse azioni e seguendo delle precise strategie, al fine di ottenere uno steering realistico. \newline
I risultati attualmente raggiungibili permettono di apprezzare la rilevanza dei vari fattori analizzati, ma quella che si ha di fronte durante il decorrere della simulazione è ancora una modesta approssimazione della realtà. \newline
La trattazione è stata articolata in quattro capitoli. Nel primo viene fatta una panoramica sul problema del comportamento delle folle discutendo le teorie socio-psicologiche su cui esso si basa, gli avvenimenti tragici che purtroppo l'hanno contraddistinto e i vari approcci presenti in letteratura per affrontarlo. \newline
Nel secondo capitolo vengono spiegate le scelte effettuate in fase di analisi, progettazione e implementazione, per definire le varie entità introdotte compatibilmente con la struttura e in relazione al meta-modello alla base di Alchemist.
Nel terzo capitolo sono raccolti alcuni esempi di simulazione, ognuno dei quali concerne un fenomeno particolarmente significativo nell'ambito dell'evacuazione di folle. \newline
Nel quarto capitolo, infine, si traggono le conclusioni sui contributi offerti da questo lavoro e si consigliano delle possibili direzioni di sviluppo future con cui è possibile ampliarlo.